\documentclass[
	% -- opções da classe memoir --
	12pt,				% tamanho da fonte
	openright,			% capítulos começam em pág ímpar (insere página vazia caso preciso)
	oneside,			% para impressão em apenas anverso. Oposto a twoside
	%twoside,			% para impressão em verso e anverso. Oposto a oneside
	a4paper,			% tamanho do papel. 
	% -- opções da classe abntex2 --
	%chapter=TITLE,		% títulos de capítulos convertidos em letras maiúsculas
	%section=TITLE,		% títulos de seções convertidos em letras maiúsculas
	%subsection=TITLE,	% títulos de subseções convertidos em letras maiúsculas
	%subsubsection=TITLE,% títulos de subsubseções convertidos em letras maiúsculas
	% -- opções do pacote babel --
	english,			% idioma adicional para hifenização
	francais,			% idioma adicional para hifenização
	spanish,			% idioma adicional para hifenização
	brazil				% o último idioma é o principal do documento
	]{abntex2}

% Evita linhas orfãs e viúvas
\widowpenalty=10000
\clubpenalty=10000

\usepackage{lmodern}			% Usa a fonte Latin Modern
\usepackage[T1]{fontenc}		% Selecao de codigos de fonte.
\usepackage[utf8]{inputenc}		% Codificacao do documento (conversão automática dos acentos)
\usepackage{lastpage}			% Usado pela Ficha catalográfica
\usepackage{indentfirst}		% Indenta o primeiro parágrafo de cada seção.
\usepackage{color}				% Controle das cores
\usepackage{graphicx}			% Inclusão de gráficos
\usepackage{microtype} 			% para melhorias de justificação
\usepackage{lipsum}				% para geração de dummy text
\usepackage[brazilian,hyperpageref]{backref}	% Paginas com as citações na bibl
\usepackage[alf]{abntex2cite}					% Citações padrão ABNT
\usepackage{graphicx}
\usepackage{tikz}
\usetikzlibrary{shapes,arrows,chains}
\usepackage[]{mcode}
\usepackage{multirow}
\usepackage{array}
\usepackage{todonotes}
\usepackage{longtable}
\usepackage{rotating}
\usepackage{caption}
\usepackage{pbox}
\usepackage{pdfpages}
\usepackage{float}
\usepackage{tikz}
\usepackage{circuitikz}	
\usetikzlibrary{babel}	% Necessário para o funcionamento do circuitikz

\usepackage[brazil]{babel}		% idiomas
\addto\captionsbrazil{
	%% ajusta nomes padroes do babel
	\renewcommand{\bibname}{Refer\^encias Bibliogr\'aficas}
	\renewcommand{\indexname}{\'Indice Remissivo}
	\renewcommand{\listfigurename}{Lista de Figuras}
	\renewcommand{\listtablename}{Lista de Tabelas}
	\renewcommand{\listadesiglasname}{Lista de Abreviaturas e Siglas}
	%% ajusta nomes usados com a macro \autoref
	\renewcommand{\pageautorefname}{p\'agina}
	\renewcommand{\sectionautorefname}{se{\c c}\~ao}
	\renewcommand{\subsectionautorefname}{subse{\c c}\~ao}
	\renewcommand{\paragraphautorefname}{par\'agrafo}
	\renewcommand{\subsubsectionautorefname}{subse{\c c}\~ao}
}

% ---
% Configurações do pacote backref
% Usado sem a opção hyperpageref de backref
\renewcommand{\backrefpagesname}{Citado na(s) página(s):~}
% Texto padrão antes do número das páginas
\renewcommand{\backref}{}
% Define os textos da citação
\renewcommand*{\backrefalt}[4]{
	\ifcase #1 %
		Nenhuma citação no texto.%
	\or
		Citado na página #2.%
	\else
		Citado #1 vezes nas páginas #2.%
	\fi}%
% ---

\definecolor{blue}{RGB}{0,114,189}
\definecolor{orange}{RGB}{217,83,25}
\definecolor{yellow}{RGB}{237,177,32}
\definecolor{purple}{RGB}{126,47,142}
\definecolor{green}{RGB}{119,172,48}
\definecolor{lightBlue}{RGB}{77,190,238}
\definecolor{red}{RGB}{162,20,47}
\definecolor{black}{RGB}{0,0,0}

% informações do PDF
\makeatletter
\hypersetup{
     	%pagebackref=true,
		pdftitle={\@title}, 
		pdfauthor={\@author},
    	pdfsubject={\imprimirpreambulo},
	    pdfcreator={LaTeX with abnTeX2},
		pdfkeywords={abnt}{latex}{abntex}{abntex2}{trabalho acadêmico}, 
		colorlinks=true,	% false: boxed links; true: colored links
    	linkcolor=black,	% color of internal links
    	citecolor=black,	% color of links to bibliography
    	filecolor=black,	% color of file links
		urlcolor=black,
		bookmarksdepth=4
}
\makeatother

% --- 
% Espaçamentos entre linhas e parágrafos 
% --- 
% O tamanho do parágrafo é dado por:
\setlength{\parindent}{1.3cm}
% Controle do espaçamento entre um parágrafo e outro:
\setlength{\parskip}{0.2cm}  % tente também \onelineskip


\titulo{Modelagem de Sistemas Não Lineares de Áudio Através de Espaço de Estados e Filtros Digitais Wave}
\autor
{
	UNIVERSIDADE FEDERAL DO RIO GRANDE DO SUL\\
	ESCOLA DE ENGENHARIA\\
	DEPARTAMENTO DE ENGENHARIA ELÉTRICA\\
	\vspace*{4\baselineskip} 
	MATHEUS OLIVEIRA DA SILVA
}
\local{Porto Alegre}
\data{2017}
\orientador{Prof. Dr. Adalberto Schuck Jr.}
\coorientador{}
\instituicao{}
\preambulo{Projeto de Diplomação apresentado ao Departamento de Engenharia Elétrica da Escola de Engenharia da Universidade Federal do Rio Grande do Sul, como requisito parcial para Graduação em Engenharia Elétrica}

\makeindex
\begin{document}
\selectlanguage{brazil}
\frenchspacing 

\imprimircapa
\imprimirfolhaderosto*

%\begin{fichacatalografica}
%	\includepdf{fichaCatalog.pdf}
%\end{fichacatalografica}

%=========================================================================
% FOLHA DE APROVAÇÃO
%=========================================================================

\begin{folhadeaprovacao}
	\begin{center}
		{\ABNTEXchapterfont\large{MATHEUS OLIVEIRA DA SILVA}}
		
		\vspace*{\fill}
		\begin{center}
			\ABNTEXchapterfont\bfseries\Large\imprimirtitulo
		\end{center}
		
		\vspace*{\fill}
		\hspace{.45\textwidth}
		\begin{minipage}{.5\textwidth}
			\imprimirpreambulo
		\end{minipage}%
	\end{center}
	
	\assinatura{\textbf{\imprimirorientador} \\ Orientador - UFRGS} 
	\todo{Atualizar Chefe do Departamento}
	\assinatura{\textbf{Prof. Dr. Ály Ferreira Flores Filho} \\ Chefe do Departamento de Engenharia Elétrica (DELET) - UFRGS}
	
	\todo{Atualizar data da apresentação}
	\begin{center}
		Aprovado em 15 de Janeiro de 2018.
	\end{center}
	
	BANCA EXAMINADORA	
	\assinatura{\textbf{Banca 1} \\ UFRGS}
	\assinatura{\textbf{Banca 2} \\ UFRGS}
	\assinatura{\textbf{Banca 3} \\ UFRGS}
\end{folhadeaprovacao}

%=========================================================================
% DEDICATÓRIA
%=========================================================================

\begin{dedicatoria}
	\vspace*{\fill}
	\centering
	\noindent
	\textit{tbd} \vspace*{\fill}
\end{dedicatoria}

%=========================================================================
% AGRADECIMENTOS
%=========================================================================

\begin{agradecimentos}
tbd
\end{agradecimentos}

%=========================================================================
% EPÍGRAFE
%=========================================================================

\begin{epigrafe}
	\vspace*{\fill}
	\begin{flushright} 
		\textit{We live in a society \\ exquisitely dependent on science and technology,\\ in which hardly anyone \\ knows anything about science and technology.}\\ \vspace{\onelineskip}
		Carl Sagan, The Demon-Haunted World
	\end{flushright}
\end{epigrafe}

%=========================================================================
% RESUMOS
%=========================================================================

% resumo em português
\setlength{\absparsep}{18pt} % ajusta o espaçamento dos parágrafos do resumo
\begin{resumo}
	Distorções em sistemas de áudio causadas por não linearidades são responsáveis pela sonoridade característica de alguns estilos musicais, por isso é importante seu estudo e compreensão. Estes "defeitos" são originalmente causados por sistemas valvulados analógicos, porém estes são de difícil mobilidade e grandes consumidores de energia. Con o poder computacional disponível atualmente é possível a reprodução destes sistemas analógicos digitalmente de forma ininteligível para o ouvido humano. Assim é atraente a ideia de simular estes com o objetivo de obter sistemas mais portáteis e econômicos.

	\vspace{\onelineskip}
	\textbf{Palavras-chave}: Sistemas não lineares. Filtros Digitais Wave. Espaço de Estados.
\end{resumo}

% resumo em inglês
\begin{resumo}[Abstract]
 \begin{otherlanguage*}{english}
	
   \vspace{\onelineskip}
   \noindent 
   \textbf{Keywords}:
 \end{otherlanguage*}
\end{resumo}

%=========================================================================
% SUMÁRIOS
%=========================================================================

% inserir lista de ilustrações
\pdfbookmark[0]{\listfigurename}{lof}
\listoffigures*
\cleardoublepage

% inserir lista de tabelas
\pdfbookmark[0]{\listtablename}{lot}
\listoftables*
\cleardoublepage

% inserir lista de abreviaturas e siglas
\begin{siglas}
	\item[LIT]		\emph{Linear Invariante no Tempo}
	\item[SNL]		\emph{Sistema Não Linear}

\end{siglas}

% inserir o sumario
\pdfbookmark[0]{\contentsname}{toc}
\tableofcontents*
\cleardoublepage

\textual
%=========================================================================
% INTRODUÇÃO
	\chapter{Introdução}
%=========================================================================

Sistemas lineares invariantes no tempo (LIT) já foram amplamente estudados por autores conhecidos como \citeonline{Haykin2003} e  \citeonline{Oppenheim1997}, tanto são que esses autores já fazem parte da bibliografia básica de disciplinas de graduação. O grande atrativo para o estudo de sistemas LIT é a simplicidade com que se pode obter a saída esperada para uma entrada tendo a resposta impulsiva do sistema, já que as únicas alterações causadas por sistemas LTI são na fase e amplitude do sinal de entrada.

Sistemas não lineares (SNL) por outro lado, apresentam saídas mais complexas pois adicionam à saída do sinal componentes com frequências múltiplas às do sinal de entrada, que são conhecidas como harmônicas. De acordo com \citeonline{Zolzer2002} efeitos não lineares são usados por músicos em diversos dispositivos como microfones amplificadores e sintetizadores.

O princípio do uso de SNLs para áudio foi com a construção de amplificadores baseados em válvulas termiônicas a partir da década de 1950 como indicado por \citeonline{Ferreira2016}. O problema destes componentes é seu peso e consumo de energia, assim foi natural sua substituição por componentes semicondutores mais leves, baratos e confiáveis, porém até hoje as distorções geradas por válvulas são vistas como superiores às geradas por semicondutores por audiófilos, um estudo sobre estas diferenças foi feito por \citeonline{Hamm1973}. Para tentar emular o som gerado por estas válvulas termiônicas em sistemas semicondutores passou a ser comum a construção de pedais de efeitos que são ligados em série com o sistemas de áudio, tendo esses a vantagem de serem mais baratos e de mais fácil transporte. O próximo passo nessa evolução é o uso de sistemas digitais para a modelagem dessas não linearidades com o objetivo de facilitar ainda mais o uso dessa tecnologia, essa enfim será a proposta deste trabalho.

Para o modelamento de SNLs são comuns 3 diferentes abordagens: modelamento de caixa branca, onde se tem total conhecimento do circuito sendo modelado; caixa cinza, onde se usa algum conhecimento do circuito para a modelagem; e caixa preta, onde não é utilizado nenhuma característica do circuito para a modelagem. \citeonline{Eichas2015} propões um modelo de caixa preta onde um ruído branco é injetado no SNL a ser modelado e a saída deste é comparada com a de um sistema paramétrico que é adaptado de maneira a minimizar o erro quadrático entre ambas. 

%=========================================================================
% FUNDAMENTAÇÃO TEÓRICA
	\chapter{Fundamentação Teórica}
%=========================================================================



%=========================================================================
% METODOLOGIA EXPERIMENTAL
	\chapter{Metodologia Experimental}
%=========================================================================

%=========================================================================
% RESULTADOS E DISCUSSÕES
	\chapter{Resultados e Discussões}
%=========================================================================

%=========================================================================
% CONCLUSÃO
	\chapter{Conclusões}
%=========================================================================

%=========================================================================
% PROPOSTA DE TRABALHOS FUTUROS
	\chapter{Propostas de Trabalhos Futuros}
%=========================================================================

%=========================================================================
% REFERÊNCIAS BIBLIOGRÁFICAS
	\postextual
	\bibliography{references}
%=========================================================================

%=========================================================================
% APÊNDICES
	\begin{apendicesenv}
	\partapendices
%=========================================================================
\end{apendicesenv}

%=========================================================================
% ANEXOS
	\begin{anexosenv}
	\partanexos
%=========================================================================
\end{anexosenv}

\end{document}
